%%%%%%%%%%%%%%%%%%%%%%%%%%%%%%%%%%%%%%%%%%%%%%%%%%%%%%%%%%%%%%%%%%%%%%%
% Sample template for MIT Junior Lab Student Written Summaries
% Available from http://web.mit.edu/8.13/samplepaper/sample-paper.tex
%
% Last Updated June 20, 2004
%
% Adapted from the American Physical Societies REVTeK-4 Pages
% at http://publish.aps.org
%
% ADVICE TO STUDENTS: Each time you write a paper, start with this
%    template and save under a new filename.  If convenient, don't
%    erase unneeded lines, just comment them out.  Often, they
%    will be useful containers for information.
%%%%%%%%%%%%%%%%%%%%%%%%%%%%%%%%%%%%%%%%%%%%%%%%%%%%%%%%%%%%%%%%%%%%%%%

%%%%%%%%%%%%%%%%%%%%%%%%%%%%%%%%%%%%%%%%%%%%%%%%%%%%%%%%%%%%%%%%%%%%%%%
% PREAMBLE
% The preamble of a LaTeX document is the set of commands that precede
% the \begin{document} line.  It contains a \documentclass line
% to load the REVTeK-4 macro definitions and various \usepackage
% lines to load other macro packages.
%
% ADVICE TO STUDENTS: This preamble contains a suggested set of
%     class options to generate a ``Junior Lab'' look and feel that
%     facilitate quick review and feedback from one's peers, TA's
%     and section instructors.  Don't make substantial changes without
%     first consulting your section instructor.
%%%%%%%%%%%%%%%%%%%%%%%%%%%%%%%%%%%%%%%%%%%%%%%%%%%%%%%%%%%%%%%%%%%%%%%

\documentclass[aps,twocolumn,secnumarabic,nobalancelastpage,amsmath,amssymb,
    nofootinbib]{revtex4}
    
    % nofootinbib is another document class option that allows you to put
    % footnotes on the page where they occur rather than at the end of the
    % paper.  This makes for easier reading!
    
    % secnumarabic is a particularly nice way of identifying sections by
    % number to aid electronic review and commentary.
    
    % amsmath and amssymb are necessary for the subequations environment
    % among others
    
    \usepackage{graphics}      % standard graphics specifications
    \usepackage{graphicx}      % alternative graphics specifications
    \usepackage{longtable}     % helps with long table options
    \usepackage{url}           % for on-line citations
    \usepackage{bm}            % special 'bold-math' package
    \usepackage[utf8]{inputenc} % para poder poner tildes
                                                                 
    %%%%%%%%%%%%%%%%%%%%%%%%%%%%%%%%%%%%%%%
    %                                 %%%%%
    % And now, begin the document...  %%%%%
    %                                 %%%%%
    %%%%%%%%%%%%%%%%%%%%%%%%%%%%%%%%%%%%%%%
    
    \begin{document}
    \title{Implementación de una librería para la detección y el análisis de interacciones de partículas con CMOS}
    \author         {Darío Federico Balmaceda}
    \email          {leschatten@gmail.com}
    \homepage       {http://fisica.cab.cnea.gov.ar/particulas/html/labdpr/}
    \affiliation    {Laboratorio Detección de Partículas y Radiación. Centro Atómico Bariloche}
    \date{\today}
    
    \begin{abstract}
    % ESTO  VALE POR UN ABSTRACT
    We present a written summary template for use by MIT Junior Lab
    students, using \LaTeX and the {\bf RevTeX-4} macro package from the
    American Physical Society.  This is the standard package used in
    preparing most Physical Review papers, and is used in many other
    journals as well.  The individual summary you hand in should show
    evidence of your own mastery of the entire experiment, and possess a
    neat appearance with concise and correct English.  The abstract is
    essential.  It should briefly mention the motivation, the method and
    most important, the quantitative result with errors.  Based on those,
    a conclusion may be drawn.  The length of the paper should be no more
    than 2 double-sided pages including all figures.
    \end{abstract}
    
    \maketitle
    
    %%%%%%%%%%%%%%%%%%%%%%%%%%%%%%%%%%%%%%%%%%%%%%%%%%%%%%%%%%%%%%%%%%%%%%%%%%%%%
    \section{Introducción}
    
    %%%%%%%%%%%%%%%%%%%%%%%%%%%%%%%%%%%%%%%%%%%%%%%%%%%%%%%%%%%%%%%%%%%%%%%%%%%%%
    \section{Configuración experimental}
    
    \subsection{Sensor CMOS}
    
    %%%%%%%%%%%%%%%%%%%%%%%%%%%%%%%%%%%%%%%%%%%%%%%%%%%%%%%%%%%%%%%%%%%%%%%%%%%%5
    
    \section{Resultados}
    
    \subsection{Medición de Rayos Cósmicos}

    \subsection{Medición de los picos $K_{\alpha}$ y $K_{\beta}$}
    
    %%%%%%%%%%%%%%%%%%%%%%%%%%%%%%%%%%%%%%%%%%%%%%%%%%%%%%%%%%%%%%%%%%%%%%%%%%%%%
    \section{Experiment}
    
    %%%%%%%%%%%%%%%%%%%%%%%%%%%%%%%%%%%%%%%%%%%%%%%%%%%%%%%%%%%%%%%%%%%%%%%%%%%%%
    \section{Conclusiones}
    

    %%%%%%%%%%%%%%%%%%%%%%%%%%%%%%%%%%%%%%%%%%%%%%%%%%%%%%%%%%%%%%%%%%%%%%%%%%%%%
    \section{Referencias}
    
    \bibliography{sample-paper}
    
    \bibliographystyle{prsty}
    \begin{thebibliography}{99}
    \bibitem{melissinos1966}Melissinos, A.C., Experiments in Modern
      Physics - 1st Edition, Academic Press,  [1966]
    \bibitem{melissinos2003}Melissinos, A.C., Napolitano, J.,  Experiments in Modern
      Physics - 2nd Edition, Academic Press,  [2003]
    \bibitem{bevington2003}Bevington and Robinson, Data Reduction and
      Error Analysis for the Physical Sciences - 3rd Edition, McGraw-Hill,
      [2003]
    \bibitem{pritchard1990}Professor D. Pritchard, Personal Communication
    \end{thebibliography}
    
    
    %%%%%%%%%%%%%%%%%%%%%%%%%%%%%%%%%%%%%%%%%%%%%%%%%%%%%%%%%%%%%%%%%%%%%%%%%%%%%
    \begin{acknowledgments} FAC gratefully acknowledges Dr. Francine Brown for
    her early reviews of this manuscript.
    \end{acknowledgments}
    
    %%%%%%%%%%%%%%%%%%%%%%%%%%%%%%%%%%%%%%%%%%%%%%%%%%%%%%%%%%%%%%%%%%%%%%%%%%%%%
    \clearpage
    \appendix
    
    
    
    \section{Using \LaTeX~ Under Windows}
    For those students who would like to use a Windows platform, MiKTeX
    (pronounced \emph{mik-tech} is a freely available, implementation of
    TeX and related programs available from \url{www.miktex.org}. Note
    that MiKTeX itself runs from a command line prompt and is not terribly
    convenient.  We strongly recommend you simultaneously purchase and
    install a very nice TeX editor/shell called WinEdt, available from
    \url{www.winedt.com} for only \$30 for students. This interface is
    substantially easier than using `emacs' on Athena for writing and
    typesetting scientific papers and we encourage you to check it out.
    
    Once you've installed the above software, you will need to obtain the
    group of files listed in the next section and put them on your Windows
    machine in order to `rebuild' this document from scratch.  MIT offers
    free of charge to students (\url{http://web.mit.edu/software/win.html}
    a variety of useful software for communicating between your Windows
    machine and your Athena account.  Three packages you should obtain and
    install are:
    \begin{verbatim}
    SecureFX
    SecureCRT
    X-Win32
    \end{verbatim}
    
    If you wish to view postscript files under Windows, we
    suggest downloading and installing Ghostscript available from
    \url{www.cs.wisc.edu/~ghost}.
    
    
    
    
    
    \section{Using \LaTeX~ Under Athena}
    For students wishing to utilize MIT's Athena environment, it is also
    a simple process to create your documents.  You can use the following
    commands verbatim or tweak them to suit your own organizational system.
    
    In your home directory on Athena, create a convenient directory structure for all of your Junior Lab
    work. Type:
    \begin{verbatim}
    > mkdir ~/8.13
    > mkdir ~/8.13/papers
    > mkdir ~/8.13/papers/template
    > cd ~/8.13/papers/template
    \end{verbatim}
    Once this (or similiar) directory structure has been created, copy all
    of the files needed to compile the template from the Junior Lab locker
    into your own Athena account: Type: 
    \begin{verbatim}
    > setup 8.13
    > cp /mit/8.13/www/Samplepaper/* .  
    \end{verbatim}
    The final period above places the
    copied files into the current directory so make sure you're in the
    correct directory!  You can see where you are by typing:
    \begin{verbatim}
    > pwd
    \end{verbatim}
    The following files should now be in
    your current directory: 
    \begin{verbatim}
    sample-paper.tex
    sample-paper.bib
    sample-fig1.pdf 
    sample-fig3.pdf 
    typical-fit-plot.pdf
    \end{verbatim}
    Additional files may also have been copied but don't worry, these get
    regenerated when you build your PDF document.
    
    The `setup' command automatically
    appends to your path the location of the {\bf RevTeX-4} files.
    
    Now let's build the file (omitting the `.tex' suffix in the following steps).  
    
    %First type: {\tt > add ghostscript}.  The
    %final step in the `build' file uses the ps2pdfwr utility to convert a
    %POSTSCRIPT version of your paper into a more easily viewed PDF format.
    %This utility is in the `ghostscript' locker on Athena.  
    
    
    \begin{verbatim}
    > pdflatex sample-paper
    > bibtex sample-paper
    > pdflatex sample-paper
    > pdflatex sample-paper
    \end{verbatim}
    
    
    The repeated calls to `pdflatex' are necessary to resolve any nested
    references in the final PDf file.  The `bibtex' call reads in the
    bibliography file `sample-paper.bib' allowing citation references to
    be resolved.
    
    {\bf Remember to {\tt ispell -t filename.tex} to perform a \LaTeX
    safe spell check before handing in your paper!}
    
    \subsection{Useful
    Athena Utilities} {\bf Drawing Programs}
    
    Students should become proficient with a simple (vector based)
    computer drawing program such as {\bf XFIG} or {\bf TGIF} on Athena.
    Every written summary should include one or two simple schematics,
    based on their initial hand sketches from their lab notebooks.
    
    {\bf Image Conversion}
    
    It is easy to  convert images from one format to another (e.g. a
    scanned jpeg or bitmap image into an pdf file for inclusion into a
    written summary).  A useful utility, available on the Sun's is
    ``imconvert''. Typing ``imconvert'' without any arguments will show
    you the accepted file types.  For example, to convert a `jpg' image
    to `pdf', one types: ``imconvert jpg:filename.jpg
    pdf:filename.pdf''.  Another command is `ps2pdf'.  
    
    
    
    % Surround figure environment with turnpage environment for landscape
    % figure
    % \begin{turnpage}
    % \begin{figure}
    % \includegraphics{}%
    % \caption{\label{}}
    % \end{figure}
    % \end{turnpage}
    
    % tables should appear as floats within the text
    %
    % Here is an example of the general form of a table:
    % Fill in the caption in the braces of the \caption{} command. Put the label
    % that you will use with \ref{} command in the braces of the \label{} command.
    % Insert the column specifiers (l, r, c, d, etc.) in the empty braces of the
    % \begin{tabular}{} command.
    % The ruledtabular enviroment adds doubled rules to table and sets a
    % reasonable default table settings.
    % Use the table* environment to get a full-width table in two-column
    % Add \usepackage{longtable} and the longtable (or longtable*}
    % environment for nicely formatted long tables. Or use the the [H]
    % placement option to break a long table (with less control than
    % in longtable).
    % \begin{table}%[H] add [H] placement to break table across pages
    % \caption{\label{}}
    % \begin{ruledtabular}
    % \begin{tabular}{}
    % Lines of table here ending with \\
    % \end{tabular}
    % \end{ruledtabular}
    % \end{table}
    
    % Surround table environment with turnpage environment for landscape
    % table
    % \begin{turnpage}
    % \begin{table}
    % \caption{\label{}}
    % \begin{ruledtabular}
    % \begin{tabular}{}
    % \end{tabular}
    % \end{ruledtabular}
    % \end{table}
    % \end{turnpage}
    
    
    \end{document}
    